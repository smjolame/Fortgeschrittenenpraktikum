\section{Theorie}
\label{sec:Theorie}

Werden mehrere Atome über Ionenbindung zusammengehalten, wird dies als Ionenkristall bezeichnet. Die genaue Kristallstruktur variiert je nach Ionenkristall.
In diesem Versuch liegt der Fokus auf Kaliumbromid, ein Ionenkristall aus Kalium- und Bromatomen. Das Kaliumbromid ist ein kubisch flächenorientierter Kristall,
jedes Ion hat also 6 direkte Nachbarn. 

\subsection{Dipole in Ionenkristallen}

Damit in einem Ionenkristall ein Dipolmoment anliegen kann, muss er mit einem Fremdmaterial dotiert werden. Dadurch, dass ein zweifach positiv geladenes Ion den Platz eines 
Kations einnimmt, kommt es zu einem Ladungsungleichgewicht, wodurch das Kation nach außen verdrängt wird und somit an seiner ursprünglichen Stelle eine Leerstelle entsteht.
Demnach kommt es zu einer Verschiebung des Ladungsschwerpunktes, wodurch ein permantentes Dipolmoment, welches zur Leerstelle hin ausgerichtet ist, entsteht. 
Bei Raumtemperatur sind die Zustandswahrscheinlichkeiten über die Boltzmann-Statistik zu beschreiben. Das Gesamtdipolmoment ist also bei Raumtemperatur null. 
Eine Umorientierung der Dipole ist dann möglich, wenn die nötige Aktivierungenergie $W$ aufgebracht werden kann, um das Gitterpotential des Kristalls zu überwinden. 
Dies kann durch thermische Energie gewährleistet werden, wobei die Relaxationszeit, also die mittlere Zeit zwischen zwei Umorientierungen durch 
\begin{equation}
    \tau(T) = \tau_0 exp \bigl( \frac{W}{k_B T}\bigr)
\end{equation}
gegeben ist. $\tau_0 $ ist dabei die charakteristische Relaxationszeit, welche ein Grenzwert für $\tau$ bei $T\shortrightarrow\infty$ angibt.