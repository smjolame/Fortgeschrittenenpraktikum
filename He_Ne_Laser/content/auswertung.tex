\section{Auswertung}
\label{sec:Auswertung}
Alle Berechnungen werden mit dem Programm \glqq Numpy" \cite{numpy}, die Unsicherheiten mit dem Modul \glqq Uncertainties" \cite{uncertainties}, die Ausgleichsrechnungen mit dem Modul \glqq Scipy" \cite{scipy} durchgeführt und die grafischen Darstellungen über das Modul \glqq Matplotlib" \cite{matplotlib} erstellt.

\subsection{Überprüfung der Stabilitätsbedingung}

Im Folgendem wird die in der Theorie \ref{sec:Theorie} gezeigte Stabilitätsbedingung für die beiden Spiegelkonstellationen überprüft.
Die aus der Messung folgenden Intensitätswerte für die Konkav-Konkav- und die Plan-Konkav-Konstellation sind jeweils den Tabellen \ref{tab:kon_kon} und \ref{tab:plan_kon} zu entnehmen. 
Zur besseren Übersicht sind die Messergebnisse in den Graphen \ref{fig:kon} und \ref{fig:plan} dargestellt. 

\begin{figure}
    \centering
    \includegraphics[width=\textwidth]{build/kon.pdf}
    \caption{Intensitätswerte konkav-konkav.}
    \label{fig:kon}
\end{figure}

\begin{figure}
    \centering
    \includegraphics[width=\textwidth]{build/plan.pdf}
    \caption{Intensitätswerte plan-konkav.}
    \label{fig:plan}
\end{figure}

Die theoretischen Stabilitätsgrenzen sind an den jeweilgen Stellen markiert. 
Es fällt auf, dass bei der Konkav-Konkav-Konstellation die Intensitätswerte für Abstände hinter der theoretischen Stabilitätsgrenze deutlich geringer sind, als für kleinere Spiegelabstände. 
So eine Tendenz ist in Abbildung \ref{fig:plan} nicht zu erkennen, da die Intensitätswerte der jeweilgen Spiegelabstände nicht bis zur Stabilitätsgrenze fortgesetzt gemessen sind. 


\subsection{Beobachtung der TEM-Moden}
 
Die beiden gemessenen Intesitätsverteilungen sind in den Abbildungen \ref{fig:TEM00} und \ref{fig:TEM01} dargestellt. Die Messwerte für die TEM$_{00}$-Mode sind der Tabelle \ref{tab:TEM00} und TEM$_{01}$ der Tabelle \ref{tab:TEM01} zu entnehmen.


\subsubsection{TEM$_{00}$-Mode}

Die Intesitätsverteilung der jeweiligen Moden ergibt nach \eqref{eqn:mode} für $m=0$ und $n=0$ eine Gaußfunktion

\begin{equation}
    I_{00}(L) = I_0 \exp{\left(-\frac{(L-a)^2}{\omega^2}\right)}
\end{equation} 

welche als Fitfunktion den Daten der TEM$_{00}$ angepasst wird.
Daraus ergeben sich die jeweilgen Parameter zu
\begin{align*}
    I_0 &= \SI{18.0(1)}{\micro\watt}, \\
    a &= \SI{0.109(033)}{\milli\m} \\
    \text{und} \: \omega &= \SI{7.25(05)}{\milli\m}.
\end{align*}

\begin{figure}
    \centering
    \includegraphics[width=\textwidth]{build/TEM_00.pdf}
    \caption{Intensitätswerte der TEM$_{00}$-Mode.}
    \label{fig:TEM00}
\end{figure}


\subsubsection{TEM$_{01}$-Mode}

Erneut wird gemäß \eqref{eqn:mode} eine Funktion in die Intesitätsverteilung gefittet. Diesemal ergibt sich für $n=0$ und $m=1$ 
eine Fitfunktion gemäß

\begin{equation}
    I_{01}(L) = \frac{8L^2}{\omega^2} \exp{\left(-\frac{(L-a)^2}{\omega^2} \right)} .
\end{equation}

Aus den Messwerten ergben sich dann die Parameter zu

\begin{align*}
    I_0 &= \SI{1.628(30)}{\micro\watt}, \\
    a &= \SI{0.33(8)}{\milli\m} \\
    \text{und} \: \omega &= \SI{6,92(5)}{\milli\m}.
\end{align*}

\begin{figure}
    \centering
    \includegraphics[width=\textwidth]{build/TEM_01.pdf}
    \caption{Intensitätswerte der TEM$_{01}$-Mode.}
    \label{fig:TEM01}
\end{figure}

\subsection{Untersuchung der Polarisation}

Die Messergebnisse sind der Tabelle \ref{tab:polar} zu entnehmen. Diese Daten sind in Abbildung \ref{fig:polar} aufgetragen. In diese Daten wird gemäß dem Gesetz von Malus \eqref{eqn:pol} 
\begin{equation}
    I = I_0 \cdot \cos(\phi - \phi_0)^2
\end{equation}
eine Ausgleichsfunktion gefittet. 
Es ergeben sich die Parameter zu
\begin{align*}
    \phi_0 & = \num{1.5038(27)} \: \text{rad} \\
    \text{und} \: I_0 & = \SI{2.850(9)}{\micro\watt}.
\end{align*}



\begin{figure}
    \centering
    \includegraphics[width=\textwidth]{build/polar.pdf}
    \caption{Intensitätswerte in Abhängigkeit des Drehwinkels des Polarisationsfilters.}
    \label{fig:polar}
\end{figure}

\subsection{Bestimmung der Wellenlänge}

Es wird die Wellenlänge des verwendeten He-Ne-Laser bestimmt. Dazu wird aus den Daten der Tabelle \ref{tab:gitter} und mit Hilfe der Gleichung \eqref{eqn:lamb} die Wellenlänge berechnet. Dabei wird der Fehler beim Ablesen der Maxima-Abstände auf $\Delta L = \SI{0.05}{\centi\m}$ geschätzt. Die Berechnung des Mittelwertes, unter Berücksichtigung der Fehlerfortpflanzung, wird mit dem Python Modul \glqq Uncertainties" \cite{uncertainties} durchgeführt.
So ergibt sich eine Wellenlänge von
\begin{equation*}
    \lambda = \SI{652(16)}{\nano\m}.
\end{equation*}

%Messwerte: Alle gemessenen physikalischen Größen sind übersichtlich darzustellen.
%
%Auswertung:
%Berechnung der geforderten Endergebnisse
%mit allen Zwischenrechnungen und Fehlerformeln, sodass die Rechnung nachvollziehbar ist.
%Eine kurze Erläuterung der Rechnungen (z.B. verwendete Programme)
%Graphische Darstellung der Ergebnisse