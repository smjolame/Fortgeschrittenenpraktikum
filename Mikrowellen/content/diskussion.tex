\section{Diskussion}
\label{sec:Diskussion}

Beim Vermessen der drei Moden des Hohlleiters ist zu erkennen, dass das Maximum des Fits nicht genau mit dem Maximum der Messwerte übereinstimmt.
Daraus lässt sich folglich eine leichte Asymmetrie der gemessenen Werte erkennen, was auf eine Ungenauigkeit beim Aufnehmen der Werte hindeuten könnte.
\\
Für die Messung der Wellenlänge und Frequenz scheint es keine direkten Auffälligkeiten zu geben. Allerding wurden diese nur über die Messung von 
zwei verschiedenen Werten berechnet, wodurch eine Ungenauigkeit bei der Messung unbemerkt geblieben sein könnte. Bei der Dämpfungsmessung ist 
hingegen bei einem Vergleich mit den Herstellerangaben auffällig, dass die gemessenen Werte zwar einen vergleichbaren strukturellen Verlauf 
aufweisen, diese jedoch um einen Konstanten Wert von $\SI{14}{\decibel}$ nach unten verschoben sind. Dies lässt sich auf einen systematischen 
Fehler beim Messen zurückführen. Nach einer Korrektur der Messwerte durch Addieren der konstanten $\SI{14}{\decibel}$ stimmen die Messwerte und 
deren Verlauf sehr genau mit den Herstellerangaben überein. 
\\
Das Stehwellenverhältnis wurde über die 3-dB-Methode und über die Abschwächermethode bestimmt. Wie zu erwarten ist anhand der 3-dB-Methode zu erkennen, 
dass das Stehwellenverhältnis mit wachsender Stifttiefe zunimmt. 
Die beiden Werte 
\begin{align*}
    S_{\SI{3}{\decibel}} = \SI{10.42}{\decibel} \\
    S_{\text{abschw.}} = \SI{14.96}{\decibel} \\
\end{align*}
liegen in der selben Größenordnung, weichen jedoch um etwa $\SI{50}{\percent}$ voneinander ab. 