\section{Theorie}
\label{sec:Theorie}

Werden mehrere Atome über Ionenbindung zusammengehalten, wird dies als Ionenkristall bezeichnet. Die genaue Kristallstruktur variiert je nach Ionenkristall.
In diesem Versuch liegt der Fokus auf Kaliumbromid, ein Ionenkristall aus Kalium- und Bromatomen. Das Kaliumbromid ist ein kubisch flächenorientierter Kristall,
jedes Ion hat also 6 direkte Nachbarn. 

\subsection{Dipole in Ionenkristallen}

Damit in einem Ionenkristall ein Dipolmoment anliegen kann, muss er mit einem Fremdmaterial dotiert werden. Dadurch, dass ein zweifach positiv geladenes Ion den Platz eines 
Kations einnimmt, kommt es zu einem Ladungsungleichgewicht, wodurch das Kation nach außen verdrängt wird und somit an seiner ursprünglichen Stelle eine Leerstelle entsteht.
Demnach kommt es zu einer Verschiebung des Ladungsschwerpunktes, wodurch ein permantentes Dipolmoment, welches zur Leerstelle hin ausgerichtet ist, entsteht. 
Bei Raumtemperatur sind die Zustandswahrscheinlichkeiten über die Boltzmann-Statistik zu beschreiben. Das Gesamtdipolmoment ist also bei Raumtemperatur null. 
Eine Umorientierung der Dipole ist dann möglich, wenn die nötige Aktivierungenergie $W$ aufgebracht werden kann, um das Gitterpotential des Kristalls zu überwinden. 
Dies kann durch thermische Energie gewährleistet werden, wobei die Relaxationszeit, also die mittlere Zeit zwischen zwei Umorientierungen durch 

\begin{equation}
    \tau(T) = \tau_0 \text{exp} \bigl( \frac{W}{k_B T}\bigr)
\end{equation}
gegeben ist. $\tau_0 $ ist dabei die charakteristische Relaxationszeit, welche ein Grenzwert für $\tau$ bei $T -> \infty$ angibt.

Wirkt ein äußeres elektrisches Feld auf den Ionenkristall, so richten sich die Dipole entlang der E-Feldlinien aus. Dafür muss die Zeit der Dipole im E-Feld 
groß gegenüber der Relaxationszeit sein. Wird bei einer bestehenden Ausrichtung der Dipole das System hinreichend abgekühlt, wird deren Ausrichtung praktisch eingefroren, sie bleibt 
also auch nach Abschalten des E-Feldes bestehen. Dies lässt sich dadurch erklären, dass bei geringen Temperaturen nicht mehr die nötige Aktivierungenergie $W$ durch Thermische Energie 
aufgebracht werden kann. Wird darauf hin das System wieder mit einer konstanten Heizrate erwärmt, können die Dipole nach und nach wieder in ihre anfängliche Ausrichtung relaxieren. 
Dadurch tritt ein Relaxationsstrom auf, der durch die zeitliche Änderung der Polarisation zustande kommt.

\subsection{Herleitung des Polarisationsstroms über Polarisationsansatz}

Die makroskopische Polarisation kann für hohe Temperaturen, dh. $p E << k_B T $ durch 

\begin{equation}
    P = \frac{N p^2 E}{3 k_B T} := y
\end{equation}

genähert werden. Dabei ist $N$ die Dichte der Dipole pro Volumen. $y(T_p)$ ist der Anteil der ausgerichteten Dipole, der mit der temperaturabhängigen Stromdichte $j(T)$ über 

\begin{equation}
    j(T) = y(T_p) p \frac{\text{d}N}{\text{d}t}
\end{equation}

zusammenhängt, wobei $\frac{\text{d}N}{\text{d}t}$ die Relaxationsrate ist, die durch $- \frac{N}{\tau(T)}$ beschreiben werden kann, mit 

\begin{equation}
    N = N_p \text{exp} \Biggl( -\frac{1}{b} \int_{T_0}^{T} \frac{\text{d}t'}{\tau(T')} \Biggr).    
\end{equation}

Die Zahl $N_p$ beschreibt die zum Beginn des Heizprozesses bestehenden ausgerichteten Dipole pro Volumen und Heizrate $b := \frac{dT}{dt}$. 
Damit lässt sich insgesamt der Dipolarisationsstrom durch 

\begin{equation}
    j(T) = \frac{p^2 E N_p}{3 k_B T_p \tau_0} \text{exp} \Biggl(-\frac{1}{b} \int_{T_0}^{T} \text{exp} \Bigl( - \frac{W}{k_B T'} \Bigr) \text{d}T' \Biggr)\cdot 
    \text{exp} \Bigl( - \frac{W}{k_B T'} \Bigr)
\end{equation}

ausdrücken. 

Bei niedrigen Temperaturen zu Anfang des Heizprozesses fließt kein signifikanter Strom, es kann also durch die Näherung 

\begin{equation}
    \int_{T_0}^{T} \text{exp} \Bigl( - \frac{W}{k_B T'} \Bigr) \approx 0 
\end{equation}

die Stromdichte in diesem Bereich als 

\begin{equation}
    j(T) = \frac{p^2 E N_p}{3 k_B T_p \tau_0} \text{exp} \Bigl( - \frac{W}{k_B T'} \Bigr)
\end{equation}

annähern.

\subsection{Herleitung des Polarisationsstroms über Stromdichte}

Der Polarisationsstrom kann im Allgemeinen durch 

\begin{equation}
    j(T) = -\frac{dP(t)}{dt} = \frac{P(t)}{\tau(T)}
\end{equation}

beschrieben werden. Mit der zeitabhängigen Polarisation 

\begin{equation}
    P(t) = P_0 \text{exp} \Bigl( - \frac{t}{\tau(T)} \Bigr)
\end{equation}

lässt die Gleichung, zusammen mit der Zeit $t$ in Integraldarstellung, als 

\begin{equation}
    j(T) = \frac{P_0}{\tau} \text{exp} \Bigl( - \int_0^t \frac{\text{d}t}{\tau(T)} \Bigr)    
\end{equation}, und mit der bereits bekannten Definition der Heizrate $b := \frac{dT}{dt}$ als 

\begin{equation}
    j(T) = \frac{P_0}{\tau} \text{exp} \Biggl( - \int_{T_0}^T exp \Bigl( -\frac{W}{k_B T'} \Bigr) \text{d}T' \Biggr)
\end{equation}

ausdrücken.