\section{Auswertung}
\label{sec:Auswertung}
Alle Berechnungen werden mit dem Programm \glqq Numpy" \cite{numpy}, die Unsicherheiten mit dem Modul \glqq Uncertainties" \cite{uncertainties}, die Ausgleichsrechnungen mit dem Modul \glqq Scipy" \cite{scipy} durchgeführt und die grafischen Darstellungen über das Modul \glqq Matplotlib" \cite{matplotlib} erstellt.


\subsection{Untersuchung des Hohleiterfeldes}

Die vom Oszillografen abgelesenen Werte sind in Tabelle \ref{tab:mod} einsehbar. 
Aus diesen Werten lässt sich ein Modus-Diagramm erstellen. Dazu werden jeweils in die Daten eine Funktion gemäß einer Parabel
\begin{equation}
    f(U) = aU^2+bU+c
\end{equation}
gelegt. 
So ergeben sich die in Abbildung \ref{fig:mode} dargestellten Moden. Die aus dem Fit erhaltenen Parameter lauten für den jeweils i-ten Modus:
\begin{align*}
    a_1 & = \SI{-0.04}{\per\square\volt} \\
    b_1 & = \SI{17.84}{\per\volt} \\
    c_1 & = \num{-1982.4} \\
    a_2 & = \SI{-0.0445}{\per\square\volt} \\
    b_2 & = \SI{ 12.2945}{\per\volt} \\
    c_2 & = \num{-840.7955} \\
    a_3 & = \SI{-0.0341}{\per\square\volt} \\
    b_3 & = \SI{5.8253}{\per\volt} \\
    c_3 & = \num{-242.8220}
\end{align*}


\begin{figure}
    \centering
    \includegraphics[width=\textwidth]{build/mode.pdf}
    \caption{Modus-Diagramm der drei Moden.}
    \label{fig:mode}
\end{figure}



\begin{table}
    \centering
    \caption{Daten von drei verschiedenen Moden.}
    \label{tab:mod}
    \sisetup{table-format = 1.2}
    \renewcommand{\arraystretch}{2} 
    \begin{tabular}{|l c| c| c| c| r|}
        \hline
         & & 1. Modus & 2. Modus & 3. Modus & \\
        \hline
        Reflektorspannung & $U_0$ & 220 & 140 & 85 & Volt \\
         & $U_1$ & 210 & 125 & 72 & \\
         & $U_2$ &  236 & 151 & 99 & \\
        Amplitude (y) & $A_0$ & 6.4 & 7.35 & 6.2 & Teilstriche \\
        Frequenz & $f_0$ & 9010 & 9013 & 9018 & $\si{\mega\hertz}$\\

        \hline

    \end{tabular}
    \renewcommand{\arraystretch}{1} 
\end{table}


Aus den Werten der Tabelle \ref{tab:elek} lassen sich die Bandbreite 
\begin{equation}
    B = f_c - f_b
\end{equation}
und die Abstimm-Empfindlichkeit 
\begin{equation}
    A = \frac{B}{V_c-V_b}
\end{equation}
zu 
\begin{equation*}
    B = \SI{78}{\mega\hertz}
\end{equation*}
und 
\begin{equation*}
    A = \SI{5.2}{\mega\hertz\per\volt}
\end{equation*}
bestimmen.

\begin{table}
    \centering
    \caption{Reflektorspannungen in Abhängigkeit der drei Kurvenpositionen.}
    \label{tab:elek}
    \sisetup{table-format = 1.2}
    \renewcommand{\arraystretch}{2} 
    \begin{tabular}{|l |  c| c| c| r|}
        \hline
         &  a) & b) & c) & \\
        \hline
        Reflektorspannung &  85 & 75 & 90 & Volt \\

        Frequenz & 9018 & 8977 & 9055 & $\si{\mega\hertz}$\\

        \hline

    \end{tabular}
    \renewcommand{\arraystretch}{1} 
\end{table}


\subsection{Frequenz, Wellenlänge und Dämpfung}

Aus den Werten der Tabelle \ref{tab:wel} lässt sich die Wellenlänge über den zweifachen Abstand benachbarter Minima berechnen.
Über den Mittelwert dieser Ergebnisse ergibt sich eine Wellenlänge von
\begin{equation}
    \lambda_h = \SI{48.9(1)}{\milli\m} .
\end{equation}

Nach \eqref{eqn:f} lässt sich die Frequenz zu
\begin{equation}
    f = \SI{8983(13)}{\mega\hertz}
\end{equation}
bestimmen.


% \begin{table}
%     \centering
%     \caption{}
%     \label{tab:plan_kon}
%     \sisetup{table-format = 4}
%     \begin{tabular}{ l | S S S S[table-format = 1.2] S}
%         \toprule
%         & $V_{0}$ & $V_{1}$ & $V_{2}$ & $A_{0}$ & $f_{0}$  \\
%         \midrule
%         \text{1. Modus}  & 220 & 210 & 236 & 6.50 & 9010 \\
%         \text{2. Modus}  & 140 & 125 & 151 & 7.35 & 9013 \\
%         \text{3. Modus}  & 85 & 71 & 99 & 6.20 & 9018 \\

%         \bottomrule

%     \end{tabular}
% \end{table}


\begin{table}
    \centering
    \caption{Frequenz und Orte der Minima.}
    \label{tab:wel}
    \sisetup{table-format = 2.1}
    \begin{tabular}{S[table-format = 4] S S S}
        \toprule
         & \multicolumn{3}{c}{Ort des Minimums $x \mathbin{/} \si{\milli\m}$} \\ 
         \cmidrule(lr){2-4}
        $f \mathbin{/} \si{\mega\hertz}$ &  $x_{1}$ & $x_{2}$ & $x_{3}$ \\
        \midrule
        9006 & 48.1 & 72.5 & 97.0 \\
        \bottomrule

    \end{tabular}
\end{table}


Die Ergebnisse der Dämpfungsmessungen sind in Tabelle \ref{tab:daempf} dargestellt. Aus den Werten der Herstellerangaben lässt sich die 
Eichkurve der Dämpfung bestimmen. Als Fit-Funktion wird dabei eine Parabel der Form
\begin{equation}
    P(x) = ax^2 + bx + c 
\end{equation}
verwendet.
Das Ergebnis dieser Ausgleichsrechnung ist in Abbildung \ref{fig:daempf_nah} dargestellt.
Die Parameter der Ausgleichsrechnung ergeben sich zu:
\begin{align}
    a &= \SI{1.7598(1515)}{\decibel\per\square\milli\m} \\
    b &= \SI{0.3299(3570)}{\decibel\per\milli\m} \\
    c &= \SI{0.0516(1881)}{\decibel}
\end{align}

\begin{figure}
    \centering
    \includegraphics[width=\textwidth]{build/daempf_nah.pdf}
    \caption{Eichkurve nach Hersteller in Form einer Fit-Parabel.}
    \label{fig:daempf_nah}
\end{figure}

\begin{table}
    \centering
    \caption{Daten der Dämpfungsmessungen.}
    \label{tab:daempf}
    \sisetup{table-format = 1.2}
    \begin{tabular}{S[table-format = 2] S S }
        \toprule
        \text{Dämpfung} $ P \mathbin{/} \si{\decibel}$ & \text{Mikrometereinstellung} / $\si{\milli\m}$ &  \text{Herstellerangabe} / $\si{\milli\m}$ \\
 
        \midrule
        0 & 2.69 & 0.00  \\
        2 & 2.93 & 0.90 \\
        4 & 3.11 & 1.45 \\
        6 & 3.27 & 1.75 \\
        8 & 3.44 & 2.03 \\
        10 & 3.58 & 2.28 \\

        \bottomrule

    \end{tabular}
\end{table}

Werden in diesen Plot die Dämpfungen in Abhängigkeit der gemessen Mikrometereinstellungen aufgetragen, so liegen alle Messpunkte in einem 
konstanten Abstand unterhalb der Herstellerangabe. Über eine Korrektur von $\SI{14}{\decibel}$ nach oben lässt sich erkennen, dass die gemessen Werte gut dem Verlauf der Parabel folgen.

Die Messwerte und die verschobenen Werte sind in Abbildung \ref{fig:daempf_weit} einzusehen und zur besseren Übersicht sind die Korrigierten Werte nocheinmal in Tabelle \ref{tab:daempf_korr} aufgeführt.

\begin{figure}
    \centering
    \includegraphics[width=\textwidth]{build/daempf_weit.pdf}
    \caption{Messwerte und auf die Eichkurve verschobenen Werte.}
    \label{fig:daempf_weit}
\end{figure}


\begin{table}
    \centering
    \caption{Daten der Dämpfungsmessungen nach Korrektur.}
    \label{tab:daempf_korr}
    \sisetup{table-format = 1.2}
    \begin{tabular}{S[table-format = 2] S S }
        \toprule
        \text{Dämpfung} $ P \mathbin{/} \si{\decibel}$ & \text{Mikrometereinstellung} / $\si{\milli\m}$ &  \text{Herstellerangabe} / $\si{\milli\m}$ \\
 
        \midrule
        14 & 2.69 & 2.72 \\
        16 & 2.93 & 2.92 \\
        18 & 3.11 & 3.10 \\
        20 & 3.27 & 3.27 \\
        22 & 3.44 & 3.44 \\
        24 & 3.58 & 3.60 \\

        \bottomrule

    \end{tabular}
\end{table}


\subsection{Stehwellen-Messungen}

Die Messergebnisse der SWR-Meter Methode und der 3-\si{\decibel}-Methode sind in den Tabellen \ref{zab:swr} und \ref{tab:3db} aufgetragen.



Aus den Ergebnissen der 3-\si{\decibel}-Methode lässt sich dann über \eqref{eqn:3db} das SWR zu
\begin{equation*}
    S = 10.42
\end{equation*}
berechnen.

Die gemessenen Werte der Abschwächer-Methode sind 
\begin{equation*}
    A_1 = \SI{20.0}{\decibel}
\end{equation*}
und
\begin{equation*}
    A_2 = \SI{43.5}{\decibel} .
\end{equation*}
Mit diesen Werten kann über den Zusammenhang \eqref{eqn:abschw} das SWR zu
\begin{equation}
    S = 14.96
\end{equation}
bestimmt werden.


\begin{table}
    \centering
    \caption{Daten der SWR-Meter Methode.}
    \label{tab:swr}
    \sisetup{table-format = 1.2}
    \begin{tabular}{S[table-format = 1] S }
        \toprule
        \text{Sondentiefe} / $\si{\milli\m}$ & \text{Dämpfung} / $\si{\decibel}$ \\
 
        \midrule
        0 & 1.00 \\
        3 & 1.11 \\
        5 & 1.55 \\
        7 & 3.20 \\
        9 & 9.50 \\

        \bottomrule

    \end{tabular}
\end{table}




\begin{table}
    \centering
    \caption{Daten der 3-dB Methode.}
    \label{tab:3db}
    \sisetup{table-format = 2.1}
    \begin{tabular}{S S S S}
        \toprule
         {$d_1 \mathbin{/} \si{\milli\m}$} & {$d_2 \mathbin{/} \si{\milli\m}$} & {$\text{1. Min} \mathbin{/} \si{\milli\m}$}  &  {$\text{2. Min} \mathbin{/} \si{\milli\m}$} \\
 
        \midrule
        74.0 & 72.5  & 108.9 & 84.5 \\
        \bottomrule

    \end{tabular}
\end{table}



%Messwerte: Alle gemessenen physikalischen Größen sind übersichtlich darzustellen.
%
%Auswertung:
%Berechnung der geforderten Endergebnisse
%mit allen Zwischenrechnungen und Fehlerformeln, sodass die Rechnung nachvollziehbar ist.
%Eine kurze Erläuterung der Rechnungen (z.B. verwendete Programme)
%Graphische Darstellung der Ergebnisse