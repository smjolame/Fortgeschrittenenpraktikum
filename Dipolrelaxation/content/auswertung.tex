\section{Auswertung}
\label{sec:Auswertung}
Alle Berechnungen werden mit dem Programm \glqq Numpy" \cite{numpy}, die Unsicherheiten mit dem Modul \glqq Uncertainties" \cite{uncertainties}, die Ausgleichsrechnungen mit dem Modul \glqq Scipy" \cite{scipy} durchgeführt und die grafischen Darstellungen über das Modul \glqq Matplotlib" \cite{matplotlib} erstellt.

Die Heizraten der jeweiligen Temperaturverläufe, wie sie in den Tabellen [] und [] zu sehen ist, lassen sich durch 

\begin{equation}
    \Delta b_i = \frac{T_i - T_{i-1}}{\SI{60}{s} }
\end{equation}

darstellen. Daraus lassen sich die beiden mittleren Heizraten 

\begin{equation}
    
\end{equation}

berechnen. 

Außerdem wird ein Untergrundstrom identifiziert, der über einen Exponetial-Fit der Form 
\begin{equation}
    f(x) = a e^{bx} + c
\end{equation}

beschrieben wird. Dieser Untergrund wird dann von den Werten der beiden Messreihen subtrahiert, damit ein charakteristischer Verlauf des Relaxationsstroms sichtbar wird. 

\subsection{Aktivierungenergie Methode 1}

Durch den in Gleichung [] gezeigten Zusammenhang können anhand der beiden Messreihen dann die Aktivierungenergien $W$ errechnet werden. Diese ergeben sich zu 

\begin{equation}
    
\end{equation}


\subsection{Aktivierungenergie Methode 2}

Eine weitere Möglichkeit, aus den Datensätzen einen Wert für die Aktivierungsenergie zu bestimmen, wird im Abschnitt [] gezeigt. Dazu wird eine lineare Regression der Form 

\begin{equation}
    F(T) = \frac{m}{T} + c
\end{equation}

durchgeführt, was die Gleichung [] wiederspiegelt. Damit werden die Aktivierungenergien beider Messreihen zu 

\begin{equation}
    
\end{equation}

bestimmt. ...

\subsection{Charakteristische Relaxationszeit}

Die charakteristische Relaxationszeit kann über die Aktivierungenergie bestimmt werden. Mit den aufgenommenen Daten kann die Gleichung [] verwendet werden, um mit den jeweils 
maximalen Temperaturen $\T_{max,i}$ jweils $\tau_0$ zu 

\begin{equation}
    
\end{equation}
zu bestimmen. 

Es lässt sich daraus eine Abhängigkeit der Relaxationszeit von der Aktivierungsenergie und der charakteristischen Relaxationszeit beschreiben, die im Plot [] sichtbar ist. Dabei 
werden hier die Werte der Integrationsmethode verwendet und es können die Verläufe der beiden Messreihen verglichen werden. 

