\section{Diskussion}
\label{sec:Diskussion}
 

In Tabelle \ref{tab:lande_lit} sind die Landé-Faktoren und ihrer Abweichungen zu den jeweiligen Literaturwerten angegeben. Die Abweichungen sind bis auf den Landé-Faktor bei blauem $\sigma$-Licht recht gering. Es ist zu beachten, dass für den Literaturwert dieses Landé-Faktors, zwei Werte existieren ($g=1.5, g=2$). Dies liegt an einem druch den Doppler-Effekt verursachten Überlapp der Linien. Desshalb wird für den Literaturwert der Mittelwert beider angenommen. 

\begin{table}
    \centering
    \caption{Landé-Faktoren und ihre Abweichungen zu den Literaturwerten.}
    \label{tab:lande_lit}
    \sisetup{table-format = 1.2}
    \begin{tabular}{S S S S S}
        \toprule
        {$\lambda / \si{\nano\meter}$ }& {$\text{Polarisierung}$} & {$\text{Literaturwert}$} & {$\text{Landé-Faktor}$ }& {$\text{Abw.} \mathbin{/} \si{\percent}$} \\
        \midrule
        480   &  {$\pi $}    & 0.5 & \num{0.58(05)} &   \num{5(5)}  \\
        480   &  $\sigma$   & 1.75 & \num{1.83(08)}  &  \num{16(10)}  \\   
        644   &  $\sigma$      & 1 & \num{0.99(07)}   & \num{1(7)} \\         
        \bottomrule

    \end{tabular}
\end{table}

Trotz der geringen Abweichung der anderen Landé-Faktoren lässt sich die Aussagekraft der Ergebnisse in Frage stellen. Zum Beispiel sind die in Abbildung \ref{fig:blau_pi} gezeigten Linien recht unscharf. Die Abstände können also nur ungenau abgelesen werden und trotzdem ist das Ergebnis recht genau.


Darüberhinaus ist die Bestimmung der Magnetfeldstärke über den Strom ein störender Umstand. Da die verwendete Hall-Sonde nicht exat an dem Ort angebracht werden kann, an dem auch das tatsächliche B-Feld herrscht, kommt es auch da zu Ungenauigkeiten in der Messung. 
%Kurze Zusammenfassung der Ergebnisse
%-Vergleich mit Literaturwerten
%-Vergleich mit verschiedenen Messverfahren
%-bei Abweichungen mögliche Ursachen finden