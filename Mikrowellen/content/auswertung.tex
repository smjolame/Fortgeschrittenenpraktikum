\section{Auswertung}
\label{sec:Auswertung}
Alle Berechnungen werden mit dem Programm \glqq Numpy" \cite{numpy}, die Unsicherheiten mit dem Modul \glqq Uncertainties" \cite{uncertainties}, die Ausgleichsrechnungen mit dem Modul \glqq Scipy" \cite{scipy} durchgeführt und die grafischen Darstellungen über das Modul \glqq Matplotlib" \cite{matplotlib} erstellt.











\begin{table}
    \centering
    \caption{}
    \label{tab:}
    \sisetup{table-format = 1.2}
    \renewcommand{\arraystretch}{2} 
    \begin{tabular}{|l c| c| c| c| r|}
        \hline
         & & 1. Modus & 2. Modus & 3. Modus & \\
        \hline
        Reflektorspannung & $U_0$ & 220 & 140 & 85 & Volt \\
         & $U_1$ & 210 & 125 & 72 & \\
         & $U_2$ &  236 & 151 & 99 & \\
        Amplitude & $A_0$ & 6.4 & 7.35 & 6.2 & Teilstriche \\
        Frequenz & $f_0$ & 9010 & 9013 & 9018 & $\si{\mega\hertz}$\\

        \hline

    \end{tabular}
    \renewcommand{\arraystretch}{1} 
\end{table}

\begin{table}
    \centering
    \caption{}
    \label{tab:}
    \sisetup{table-format = 1.2}
    \renewcommand{\arraystretch}{2} 
    \begin{tabular}{|l |  c| c| c| r|}
        \hline
         &  a) & b) & c) & \\
        \hline
        Reflektorspannung &  85 & 75 & 90 & Volt \\

        Frequenz & 9018 & 8977 & 9055 & $\si{\mega\hertz}$\\

        \hline

    \end{tabular}
    \renewcommand{\arraystretch}{1} 
\end{table}

% \begin{table}
%     \centering
%     \caption{}
%     \label{tab:plan_kon}
%     \sisetup{table-format = 4}
%     \begin{tabular}{ l | S S S S[table-format = 1.2] S}
%         \toprule
%         & $V_{0}$ & $V_{1}$ & $V_{2}$ & $A_{0}$ & $f_{0}$  \\
%         \midrule
%         \text{1. Modus}  & 220 & 210 & 236 & 6.50 & 9010 \\
%         \text{2. Modus}  & 140 & 125 & 151 & 7.35 & 9013 \\
%         \text{3. Modus}  & 85 & 71 & 99 & 6.20 & 9018 \\

%         \bottomrule

%     \end{tabular}
% \end{table}


\begin{table}
    \centering
    \caption{}
    \label{tab:}
    \sisetup{table-format = 2.1}
    \begin{tabular}{S[table-format = 4] S S S}
        \toprule
         & \multicolumn{3}{c}{Ort des Minimums $x \mathbin{/} \si{\milli\m}$} \\ 
         \cmidrule(lr){2-4}
        $f \mathbin{/} \si{\mega\hertz}$ &  $x_{1}$ & $x_{2}$ & $x_{3}$ \\
        \midrule
        9006 & 48.1 & 72.5 & 97.0 \\
        \bottomrule

    \end{tabular}
\end{table}


\begin{table}
    \centering
    \caption{}
    \label{tab:}
    \sisetup{table-format = 1.2}
    \begin{tabular}{S[table-format = 2] S S }
        \toprule
        \text{Dämpfung} $ P \mathbin{/} \si{\decibel}$ & \text{Mikrometereinstellung} / $\si{\milli\m}$ &  \text{Herstellerangabe} / $\si{\milli\m}$ \\
 
        \midrule
        0 & 2.69 & 0.00  \\
        2 & 2.93 & 0.90 \\
        4 & 3.11 & 1.45 \\
        6 & 3.27 & 1.75 \\
        8 & 3.44 & 2.03 \\
        10 & 3.58 & 2.28 \\

        \bottomrule

    \end{tabular}
\end{table}

\begin{table}
    \centering
    \caption{}
    \label{tab:}
    \sisetup{table-format = 1.2}
    \begin{tabular}{S[table-format = 1] S }
        \toprule
        \text{Sondentiefe} / $\si{\milli\m}$ & \text{Dämpfung} / $\si{\decibel}$ \\
 
        \midrule
        0 & 1.00 \\
        3 & 1.11 \\
        5 & 1.55 \\
        7 & 3.20 \\
        9 & 9.50 \\

        \bottomrule

    \end{tabular}
\end{table}




\begin{table}
    \centering
    \caption{}
    \label{tab:}
    \sisetup{table-format = 2.1}
    \begin{tabular}{S S S S}
        \toprule
         {$d_1 \mathbin{/} \si{\milli\m}$} & {$d_2 \mathbin{/} \si{\milli\m}$} & {$\text{1. Min} \mathbin{/} \si{\milli\m}$}  &  {$\text{2. Min} \mathbin{/} \si{\milli\m}$} \\
 
        \midrule
        74.0 & 72.5  & 108.9 & 84.5 \\
        \bottomrule

    \end{tabular}
\end{table}



%Messwerte: Alle gemessenen physikalischen Größen sind übersichtlich darzustellen.
%
%Auswertung:
%Berechnung der geforderten Endergebnisse
%mit allen Zwischenrechnungen und Fehlerformeln, sodass die Rechnung nachvollziehbar ist.
%Eine kurze Erläuterung der Rechnungen (z.B. verwendete Programme)
%Graphische Darstellung der Ergebnisse