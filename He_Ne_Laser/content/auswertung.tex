\section{Auswertung}
\label{sec:Auswertung}
Alle Berechnungen werden mit dem Programm \glqq Numpy" \cite{numpy}, die Unsicherheiten mit dem Modul \glqq Uncertainties" \cite{uncertainties}, die Ausgleichsrechnungen mit dem Modul \glqq Scipy" \cite{scipy} durchgeführt und die grafischen Darstellungen über das Modul \glqq Matplotlib" \cite{matplotlib} erstellt.

\subsection{Überprüfung der Stabilitätsbedingung}

Im Folgendem wird die in der Theorie \ref{sec:Theorie} gezeigte Stabilitätsbedingung für die beiden Spiegelkonstellationen überprüft.
Die aus der Messung folgenden Intensitätswerte für die konkav-konkav- und die plan-konkav-Konstellation sind jeweils den Tabellen \ref{tab:kon_kon} und \ref{tab:plan_kon} zu entnehmen. 
Zur besseren Übersicht sind die Messergebnisse in den Graphen \ref{fig:kon} und \ref{fig:plan} dargestellt. 

\begin{figure}
    \centering
    \includegraphics[width=\textwidth]{build/kon.pdf}
    \caption{Intensitätswerte des Lasers in Abhängigkeit des Spiegelabstandes bei konkav-konkaver Konstellation.}
    \label{fig:kon}
\end{figure}

\begin{figure}
    \centering
    \includegraphics[width=\textwidth]{build/plan.pdf}
    \caption{Intensitätswerte des Lasers in Abhängigkeit des Spiegelabstandes bei plan-konkaver Konstellation.}
    \label{fig:plan}
\end{figure}

Die theoretische Stabilitätsgrenze ist an der jeweiligen Stellen markiert. Da bei konkav-konkaver Konstellation die Stabilitätsbedingung für alle Abstände erfüllt ist, lässt sich dort keine Stabilitätsgrenze festlegen. 

\subsection{Beobachtung der TEM-Moden}
 
Die beiden gemessenen Intesitätsverteilungen sind in den Abbildungen \ref{fig:TEM00} und \ref{fig:TEM01} dargestellt. Die Messwerte für die TEM$_{00}$-Mode sind der Tabelle \ref{tab:TEM00} und TEM$_{01}$ der Tabelle \ref{tab:TEM01} zu entnehmen.


\subsubsection{TEM$_{00}$-Mode}

Die Intesitätsverteilung der jeweiligen Moden ergibt nach \eqref{eqn:mode} für $m=0$ und $n=0$ eine Gaußfunktion

\begin{equation}
    I_{00}(L) = I_0 \exp{\left(-\frac{(L-a)^2}{\omega^2}\right)}
    \label{eqn:TEM00}
\end{equation} 

welche als Fitfunktion den Daten der TEM$_{00}$ angepasst wird.
Daraus ergeben sich die jeweiligen Parameter zu
\begin{align}
    \label{ali:TEM00}
    I_0 &= \SI{18.0(1)}{\micro\watt}, \\
    a &= \SI{0.11(03)}{\milli\m} \\
    \text{und} \: \omega &= \SI{7.25(05)}{\milli\m}.
\end{align}

\begin{figure}
    \centering
    \includegraphics[width=\textwidth]{build/TEM_00.pdf}
    \caption{Intensitätswerte der TEM$_{00}$-Mode mit Ausgleichsfunktion gemäß Gleichung \eqref{eqn:TEM00} und den jeweiligen Parametern \eqref{ali:TEM00}.}
    \label{fig:TEM00}
\end{figure}


\subsubsection{TEM$_{01}$-Mode}

Erneut wird gemäß \eqref{eqn:mode} eine Funktion in die Intesitätsverteilung gefittet. Diesemal ergibt sich für $n=0$ und $m=1$ 
eine Fitfunktion gemäß

\begin{equation}
    I_{01}(L) = \frac{8L^2}{\omega^2} \exp{\left(-\frac{(L-a)^2}{\omega^2} \right)} .
    \label{eqn:TEM01}
\end{equation}

Aus den Messwerten ergben sich dann die Parameter zu

\begin{align}
    \label{ali:TEM01}
    I_0 &= \SI{1.63(3)}{\micro\watt}, \\
    a &= \SI{0.33(8)}{\milli\m} \\
    \text{und} \: \omega &= \SI{6,92(5)}{\milli\m}.
\end{align}

\begin{figure}
    \centering
    \includegraphics[width=\textwidth]{build/TEM_01.pdf}
    \caption{IntensitätswerteIntensitätswerte der TEM$_{01}$-Mode mit Ausgleichsfunktion gemäß Gleichung \eqref{eqn:TEM01} und den jeweiligen Parametern \eqref{ali:TEM01}.}
    \label{fig:TEM01}
\end{figure}

\subsection{Untersuchung der Polarisation}

Die Messergebnisse sind der Tabelle \ref{tab:polar} zu entnehmen. Diese Daten sind in Abbildung \ref{fig:polar} aufgetragen. An diese Daten wird gemäß dem Gesetz von Malus \eqref{eqn:pol} 
\begin{equation}
    I = I_0 \cdot \cos(\phi - \phi_0)^2
    \label{eqn:malus}
\end{equation}
eine Ausgleichsfunktion gefittet. 
Es ergeben sich die Parameter zu
\begin{align}
    \label{ali:malus}
    \phi_0 & = \num{1.504(3)} \: \text{rad} \\
    \text{und} \: I_0 & = \SI{2.850(9)}{\micro\watt}.
\end{align}



\begin{figure}
    \centering
    \includegraphics[width=\textwidth]{build/polar.pdf}
    \caption{Intensitätswerte in Abhängigkeit des Drehwinkels des Polarisationsfilters mit Ausgleichsfunktion gemäß Gleichung \ref{eqn:malus} und den jeweiligen Parametern \eqref{ali:malus}.}
    \label{fig:polar}
\end{figure}

\subsection{Bestimmung der Wellenlänge}

Es wird die Wellenlänge des verwendeten He-Ne-Laser bestimmt. Dazu wird aus den Daten der Tabelle \ref{tab:gitter} und mit Hilfe der Gleichung \eqref{eqn:lamb} die Wellenlänge berechnet. Dabei wird der Fehler beim Ablesen der Maxima-Abstände auf $\Delta L = \SI{0.05}{\centi\m}$ geschätzt. Die Berechnung des Mittelwertes, unter Berücksichtigung der Fehlerfortpflanzung, wird mit dem Python Modul \glqq Uncertainties" \cite{uncertainties} durchgeführt. 

\begin{table}
    \centering
    \caption{Ergebnisse der Wellenlängenbestimmung.}
    \label{tab:well}
    \sisetup{table-format = 1.2}
    \begin{tabular}{S S}
        \toprule
        $\text{Gitterkonstante} \mathbin{/}\si{\per\milli\m}$ & $\lambda \mathbin{/} \si{\nano\m}$ \\
        \midrule
        \text{\nicefrac{1}{600}}& \num{635.9(7)}   \\
        \text{\nicefrac{1}{100}} & \num{700.2(11)}   \\
        \text{\nicefrac{1}{1200}}& \num{636.0(5)}   \\
        \text{\nicefrac{1}{80}}& \num{635.3(11)}   \\

        \bottomrule

    \end{tabular}
\end{table}

So ergibt sich eine Wellenlänge von
\begin{equation*}
    \lambda = \SI{652(16)}{\nano\m}.
\end{equation*}

%Messwerte: Alle gemessenen physikalischen Größen sind übersichtlich darzustellen.
%
%Auswertung:
%Berechnung der geforderten Endergebnisse
%mit allen Zwischenrechnungen und Fehlerformeln, sodass die Rechnung nachvollziehbar ist.
%Eine kurze Erläuterung der Rechnungen (z.B. verwendete Programme)
%Graphische Darstellung der Ergebnisse