\section{Auswertung}
\label{sec:Auswertung}
Alle Berechnungen werden mit dem Programm \glqq Numpy" \cite{numpy}, die Unsicherheiten mit dem Modul \glqq Uncertainties" \cite{uncertainties}, die Ausgleichsrechnungen mit dem Modul \glqq Scipy" \cite{scipy} durchgeführt und die grafischen Darstellungen über das Modul \glqq Matplotlib" \cite{matplotlib} erstellt.

Die Heizraten der jeweiligen Temperaturverläufe, wie sie in den Tabellen [] und [] zu sehen ist, lassen sich durch 

\begin{equation}
    \Delta b_i = \frac{T_i - T_{i-1}}{\SI{60}{s} }
\end{equation}

darstellen. Daraus lassen sich die beiden mittleren Heizraten 

\begin{align*}
    h_1 &=  \SI{1.360(029)}{\kelvin\per\second} \\
    h_2 &=  \SI{1.912(056)}{\kelvin\per\second} \\
\end{align*}

berechnen. 

Außerdem wird ein Untergrundstrom identifiziert, der über einen Exponetial-Fit der Form 
\begin{equation}
    f(x) = a e^{bx} + c
\end{equation}

beschrieben wird. Dieser Untergrund wird dann von den Werten der beiden Messreihen subtrahiert, damit ein charakteristischer Verlauf des Relaxationsstroms sichtbar wird. 

\subsection{Aktivierungenergie über Polarisationsansatz}

Über den Polarisationsansatz kann, wie im Abschnitt \ref{sec:pol} beschrieben, die Aktivierungenergie $W$ näherungsweise bestimmt werden.
Die so errechneten Werte für die beiden Messreihen ergeben sich zu 
\begin{align*}
    W_{\text{int},1.5} &= \SI{1.46(06)e-19}{\joule} &= \SI{0.91(04)}{\electronvolt} \\
    W_{\text{int},2}   &= \SI{9.4(12)e-20}{\joule} &= \SI{0.59(08)}{\electronvolt} \\
\end{align*}


\subsection{Aktivierungenergie über Stromdichte}

Eine weitere Möglichkeit, aus den Datensätzen einen Wert für die Aktivierungsenergie zu bestimmen, wird im Abschnitt \ref{sec:int} gezeigt. 
Dazu wird eine lineare Regression der Form 

\begin{equation}
    F(T) = \frac{m}{T} + c
\end{equation}

durchgeführt, was die Gleichung 
\begin{equation}
    \ln{\tau} = \ln(\tau_0) + \frac{W}{k_B T}
\end{equation} 
wiederspiegelt.

Damit werden die Aktivierungenergien beider Messreihen zu 

\begin{align*}
    W_{\text{apr},1.5} &= \SI{8.4(5)e-20}{\joule} &= \SI{0.524(033)}{\electronvolt} \\
    W_{\text{apr},2}   &= \SI{9.0(5)e-20}{\joule} &= \SI{0.560(033)}{\electronvolt} \\
\end{align*}

bestimmt.

\subsection{Charakteristische Relaxationszeit}

Die charakteristische Relaxationszeit kann über die Aktivierungenergie bestimmt werden. Mit den aufgenommenen Daten  und den bereits 
errechneten Heizraten kann die Gleichung 
\begin{equation}
    T^2_\text{max} = bW = \frac{\tau(T_\text{max})}{k_B} 
\end{equation} 
verwendet werden, um mit den jeweils maximalen Temperaturen $T_{max,1.5} = \SI{259.2}{\kelvin} $ und $T_{max,2} = \SI{260.5}{\kelvin} $ 
jweils $\tau_0$  für beide Methoden zu bestimmen. \\

Für die Werte des Polarisationsansatzes ergeben sich die $\tau_0$ damit zu 
\begin{align*}
    \tau_{0,1.5} &= \SI{2.5(29)e-12}{\second}\\
    \tau_{0,2}   &= \SI{1.8(21)e-12}{\second}\\ 
\end{align*}
und für den Ansatz über die Stromdichte ergibt sich 
\begin{align*}
    \tau_{0,1.5} &= \SI{1.9(32)e-13}{\second}\\
    \tau_{0,2}   &= \SI{1.4(24)e-13}{\second}\\ 
\end{align*}

Es lässt sich daraus eine Abhängigkeit der Relaxationszeit von der Aktivierungsenergie und der charakteristischen Relaxationszeit beschreiben, die im Plot [] sichtbar ist. Dabei 
werden hier die Werte der Integrationsmethode verwendet und es können die Verläufe der beiden Messreihen verglichen werden. 

