\section{Durchführung}
\label{sec:Durchführung}

%Was wurde gemessen bzw. welche Größen wurden variiert?
\subsection{Aufbau}

Für den Aufbau wird eine optische Schiene verwendet, auf der alle Bauteile montiert sind. An einem Ende befindet sich zu Anfang ein Justierlaser, mit dem der eigentliche Laser 
ausgerichtet werden kann. In aufgezählter Reihenfolge befinden sich hinter dem Justierlaser eine Blende, der erste Resonatorspiegel, das Laserrohr und der zweite Resonatorspiegel. 
Dabei ist das Laserrohr mit einem Gasgemisch aus Helium und Neon befüllt. Die Resonatorspiegel sind jeweils konkav mit einem Krümmungsradius von $r = \SI{1400}{mm} $ und lassen 
sich in ihrer Ausrichtung anhand zweier Stellschrauben variieren, so dass der Laser funktionieren kann. Hinter dem Resonator werden je nach Messung noch andere Bauteile montiert, 
die im Folgende spezifiziert werden. 

\subsection{Durchführung}

\subsubsection{Justierung des Lasers}
Zum Ausrichten des eigentlichen Lasers wird der Justierlaser verwendet. Dabei muss darauf geachtet werden, dass das Licht des Justierlasers auf ein dafür vorgesehenes Fadenkreuz
trifft. Dazu werden vorsichtig die Resonatorspiegel ausgerichtet, bis die gewünschte Einstellung erreicht ist. Nach Ausstellen des Justierlasers ist nun ein roter Laserstrahl des 
Helium Neon Lasers zu sehen. 

\subsubsection{Überprüfen der Stabilitätsbedingung}
Um die Stabilitätsbedingung zu überprüfen, wird hinter den Laser eine Photodiode plaziert, um die Intensität des Lasers bei variierendem Abstand der Resonatorspiegel zu detektieren.
Dies wird einmal für die ursprüngliche Konfiguration mit zwei konkaven Spiegeln mit einem Abstand von $\SI{63}{cm} $ bis $\SI{194}{cm} $ und für einen Plan-konkaven Resonator mit 
Abstand $\SI{50}{cm} $ bis $\SI{128}{cm} $ durchgeführt. 

\subsubsection{TEM-Moden}
Für die Beobachtung der TEM-Moden wird wieder ein konkav-konkaver Resonator verwendet. Hinter dem Auskoppelspiegel wird eine Sammellinse montiert, um die Moden
vergrößert sichtbar zu machen. Zuerst wird die $\text{TEM}_{00} $ Mode vermessen. Es wird ein Schirm zischen Laser und Diode gesetzt, um die Diode auf den Laserstrahl auszurichten. 
Die Diode wird dann senkrecht zur Ausbreitungsrichtung des Strahls verschoben, um den Intensitätsverlauf der Mode zu vermessen. 
Die gleiche Prozedur wird für die $\text{TEM}_{01} $ Mode durchgeführt, nur dass hierbei ein menschliches Haar zwischen dem Laserrohr und dem Auskoppelspiegel positioniert wird, um 
diese Mode sichtbar zu machen. 

\subsubsection{Polarisation}
Ausgehend vom Grundaufbau wird für die Bestimmung der Polarisation ein Polarisationsfilter hinter dem Auskoppelspiegel montiert. Dieser lässt sich in der Ebene senkrecht zur 
Ausbreitungsrichtung des Lasers drehen, um seine Ausrichtung zu ändern. Dies wird in $\SI{4}{\degree}$ Schritten getan, wobei jeweils die Intensität hinter dem Filter anhand einer 
Photodiode abelesen wird. 

\subsubsection{Wellenlänge}
Zur Bestimmung der Wellenlänge des Lasers werden anstelle des Polarisationsfilters verschiedene Gitter eingesetzt. Für die Gitterkonstanten werden hier $\SI{}{}$, $\SI{}{}$ und 
$\SI{}{}$ gewählt. Auf einem Schirm mit jeweils fixiertem Abstand werden die Abstände der Interferenzmaxima, die durch die Gitter zustandekommen, notiert. 
