\section{Theorie}
\label{sec:Theorie}

\subsection{Drehimpuls und magnetisches Moment eines Elektrons}

Die Drehimpulskomponenten eines Elektrons, welches an ein Atom gebunden ist, sind der Bahndrehimpuls $\vec{l}$ und der Spin $\vec{s}$.
Die Beträge lassen sich jeweils durch die Quantenzahlen $l$ und $s$ ausdrücken:


\begin{align}
    |\vec{l}| &= \sqrt{l(l+1)\hbar} \\
    |\vec{s}| &= \sqrt{s(s+1)\hbar} 
\end{align}

Dabei ist $s= \frac{1}2$ und $l$ kann Werte von $0$ bis $n-1$ annehmen, wobei n die Hauptquantelzahl ist. 
Den Drehimpulsen der Elektronen können aufgrund deren Ladungen jeweils magnetische Momente zugeordnet werden:

\begin{align}
    \vec{\mu_l} &= -\frac{\vec{\mu_B}}{\hbar}\vec{l} = -\vec{\mu_B}\sqrt{l(l+1)\vec{l_e}} \\
    \vec{\mu_s} &= -g_s\frac{\vec{\mu_B}}{\hbar}\vec{s} = -g_s\vec{\mu_B}\sqrt{s(s+1)\vec{s_e}} 
\end{align}

Hierbei ist $\vec{\mu_B}$ das Bohr'sche Magneton, welches den Wert 

\begin{equation}
    \vec{\mu_B} = \frac{1}2 e_0 \frac{\hbar}{m_0}
\end{equation}

hat, $g_s$ der Landé-Faktor mit $g_s\approx 2$ und $\vec{l_e}$ bzw. $\vec{s_e}$ die Einheitsvektoren von $\vec{l}$ und $\vec{s}$. 

\subsection{Wechselwirkungen der Drehimpulse}

Für diesen Versuch werden im Folgenden zwei Grenzfälle der Kopplung von Spin- und Bahndrehimpuls genauer Erläutert.
Bei der sogenannten j-j-Kopplung werden Atome mit hoher Masse betrachtet. Hierbei ist die Kopplung der unterschiedlichen Drehimpulse eines 
Elektrons zueinander größer, als die Kopplung zu anderen Elektronen. Der Gesamtdrehimpuls eines Elektrons ist also hier 
\begin{equation}
    \vec{j_i} = \vec{l_i} + \vec{s_i}
\end{equation} mit dem Gesamtdrehimpuls der Elektronen im Atom 
\begin{equation}
    \vec{J_i} = \sum \vec{j_i}\,.
\end{equation}

Mit der L-S-Kopplung werden Atome mit leichten Kernen, also niedriger Massezahl, betrachtet. Hierbei überwiegt die Wechselwirkung zwischen den 
einzelnen Elektronen, folglich wird der Gesamtdrehimpuls $\vec{J}$ durch 
\begin{equation}
    \vec{J} = \vec{L} + \vec{S}
\end{equation}
beschrieben, mit 
\begin{align*}
    \vec{L_i} &= \sum \vec{l_i} \\
    \vec{S_i} &= \sum \vec{s_i}
\end{align*}

Die Magnetischen Momente der aufaddierten Drehimpulse sind dann 

\begin{align*}
    \vec{\mu_L} &=  \vec{\mu_B} \sqrt{L(L+1)} \\
    \vec{\mu_S} &=  g_s \vec{\mu_B} \sqrt{S(S+1)} 
\end{align*}

\subsection{Aufpaltung der Energieniveaus}
Wird das Magnetische Moment des Gesamtdrehimpulses $\vec{J}$ durch addieren der magnetischen Momente von $\vec{S}$ und $\vec{L}$
beschrieben, also 
\begin{equation}
    \vec{\mu_J} = \vec{\mu_L} + \vec{\mu_S}\,,
\end{equation}

muss darauf geachtet werden, dass $\vec{\mu}$ nicht die selbe Richtung wie $\vec{J}$ aufweist. Folglich müssen alle Komponenten, bei denen 
$\vec{\mu}$ und $\vec{J}$ senkrecht aufeinander stehen, verschwinden. Für den Betrag der übrig bleibenden parrallelen Komponenten gilt dann 
\begin{equation}
    |\vec{\mu_J}| = g_J \mu_B \sqrt{J(J+1)}\,
\end{equation} 
wobei 
\begin{equation}
    g_J = \frac{3J(J+1) + S(S+1) -L(L+1)}{2J(J+1)}\,.
\end{equation}
Bei Bestand eines äußeren Magnetfeldes $\vec{B}$ muss aufgrund der Richtungsquantelung der Winkel zwischen $\vec{B}$ und $\vec{\mu}$ so stehen,
dass die $z-$Komponente des magnetischen Momentes $\mu_{J_{z}}$ immer ein ganzzaliges Vielfaches des Produktes aus Bohr'schem Magneton und 
Landé-Faktor ist, dh. 
\begin{equation}
    \mu_{J_{z}} = - m\cdot g_J\cdot \mu_B,\.\.\.m=[-J,J]\. ,
\end{equation}
mit der ganzzaligen Orientierungsquantenzahl $m$. Die Energie, die dem System durch das Magnetfeld hinzugefügt wird, kann damit durch 

\begin{equation}
    E_{mag} =  -\vec{\mu_J} \vec{B} = m\cdot g_J\cdot \mu_B\cdot B
\end{equation}

beschrieben werden. Es folgt die Möglichkeit der Aufpaltung der Energieniveaus in $2J+1$ Niveaus.
Auch bei bereits angeregten Energieniveaus ist diese Aufspaltung möglich, wodurch es zu weiteren Übergängen zwischen den aufgespaltenen Niveaus 
kommen kann. Dieser Effekt wird Zeeman-Effekt genannt und lässt sich durch die dadurch auftretende Aufspaltung in den Spektrallinien betrachten.

\subsection{Normaler und Anormaler Zeeman-Effekt}

Der normale Zeeman-Effekt ist im Grunde genommen ein Spezielfall, bei welchem der Gesamtspin des Atoms gleich null sein muss, wodurch der 
Landé-Faktor zu $g_J = 1$ wird. Eine Verschiebung des Energieniveaus ergibt sich damit zu 
\begin{equation}
    \Delta E = m\cdot g_J\cdot \mu_B\cdot B
\end{equation}

Dadurch entsteht also eine Aufspaltung der Energieniveaus in $2J+1$ Niveaus, wobei die Linien anhand der jeweiligen Polarisation kategorisiert
werden. Die $\sigma_-$ und $\sigma_+$ Linien bei zirkularer Polarisation und entsprechend $\Delta M \pm 1$ und die $\pi$-Linie bei linearer 
Polarisation und $\Delta M =0$. Die Energiedifferenzen der Übergänge lassen sich allein anhand von $\Delta M$ unterscheiden, weshalb eine 
Aufspaltung in drei Spektrallinien vorliegt. Die Polarisation führt dazu, dass bei einer Beobachtung die Perspektive eine essentielle Rolle spielt.
Bei einer Beobachtung entlang der Feldrichtung verschwindet die $\pi$-Linie, wobei die zirkular polarisierten $\sigma_{\pm}$-Linien bei 
Betrachtung senkrecht zur Feldrichtung linear polarisiert erscheinen. 
\\
Beim anormalen Zeeman-Effekt werden die Fälle betrachtet, bei denen der Gesamtspin der Elektronen ungleich null ist. Da jetzt $g_J$ nicht mehr 
einfach gleich eins ist, sind nun die Energiedifferenzen spinanhängig. Genauer lassen sich diese durch 
\begin{equation}
    \Delta E = (m_i g(L_i,S_i,J_i) -m_k g(L_k,S_k,J_k,))\mu_B B + E_0
\end{equation}
beschreiben. $E_0$ entspricht dabei der Energie ohne äußeres Magnetfeld und die Indices $i$ und $k$ kennzeichnen die Überganngsnivueas. 

\subsection{Landé-Faktoren}

Im Rahmen dieses Versuchs lassen sich die Landé-Faktoren anhand der Gleichung 
\begin{equation}
    \Delta E = g\mu_B B 
\end{equation} 

bestimmen. Mit

\begin{equation}
    \Delta E = \frac{\delta E}{\delta\lambda}E(\lambda)
\end{equation}

und $E(\lambda) = \frac{hc}{\lambda}$ lässt sich der Zusammenhang durch 
\begin{equation*}
    g = \delta\lambda\frac{hc}{\mu_B B \lambda^2}
    \label{eqn:lande}
\end{equation*} darstellen. 