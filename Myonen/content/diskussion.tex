\section{Diskussion}
\label{sec:Diskussion}


Die Lebensdauer von Myonen hat laut Literatur \cite{Elementarteilchenphysik} einen Wert von $\tau_\text{lit} = \SI{2.197}{\micro\s}$. 
Damit weicht der Literaturwert um $\SI{7.6(28)}{\percent}$ von dem berechneten Wert $\tau_\text{exp} = \SI{2.36(06)}{\micro\s}$ ab. Unter Beachtung der vielen Störungsquellen (Fehlsignale der PMT, Eintritt mehrerer Myonen innerhalb der Suchzeit, nicht optimale Diskriminatoren -Schwellen) ist diese Abweichung relativ gering. 

Dabei ist der Unterschied zwischen der Fit-Untergrundrate $N_U^{fit} = \num{0.9282(29472)} \: \frac{\text{Signale}}{\text{Kanal}}$ und der berechneten Untergrundrate $N_U^{exp} = \num{8.85} \: \frac{\text{Signale}}{\text{Kanal}}$ recht groß. Das wird daran liegen, dass ca. nach dem 132. Kanal nur noch sehr wenig Kanäle mit maximal einem Signal vorkommen. Wenn diese Kanäle nicht nur entweder ein oder kein Siganl gespeichert hätten, wäre der Fit seichter gegen die Lebensdauer-Achse verlaufen und der Parameter für die Untergrundrate wäre größer ausgefallen. Darüberhinaus hat der MCA nur $\num{15150}$ Stoppsignale gemessen, obwohl laut Stoppsignal-Zähler $\num{17775}$ Stoppsignale regestriert sind. Diese Differenz ist wahrscheinlich darauf zurückzuführen, dass der MCA bei schnell hintereinander eintreffenden Signalen, nur eins davon regestriert.

%Kurze Zusammenfassung der Ergebnisse
%-Vergleich mit Literaturwerten
%-Vergleich mit verschiedenen Messverfahren
%-bei Abweichungen mögliche Ursachen finden