\section{Diskussion}
\label{sec:Diskussion}


Die Lebensdauer von Myonen hat laut Literatur \cite{Elementarteilchenphysik} einen Wert von $\tau_\text{lit} = \SI{2.197}{\micro\s}$. 
Damit weicht der Literaturwert um $\SI{14.2(16)}{\percent}$ von dem berechneten Wert $\tau_\text{exp} = \SI{2.51(04)}{\micro\s}$ ab. Diese Abweichung ist unter Beachtung des einfachen Aufbaus relativ gering. 

Da die Ausgleichsrechnungen nach Abzug der vorher bestimmten Untergrundrate durchgeführt wird, ist die Untergrundrate der Ausgleichsrechnung $N_\text{U}^{\text{fit}} = \num{0.374(0795)} \: \frac{\text{Signale}}{\text{Kanal}}$ relativ gering. Dies deutet darauf hin, dass die bestimmte Untergrundrate einigermaßen passend bestimmt wird. Darüberhinaus hat der MCA nur $\num{15150}$ Stoppsignale gemessen, obwohl laut Stoppsignal-Zähler $\num{17775}$ Stoppsignale regestriert sind. Diese Differenz ist wahrscheinlich darauf zurückzuführen, dass der MCA bei schnell hintereinander eintreffenden Signalen, nur eins davon regestriert. 

Die Halbwertsbreite von $t_\text{FWHM} = \SI{14}{\nano\s}$ ist im erwarteten Bereich, da die Abklingzeit des Szintillatormaterials in dieser Größenordnung liegt. 

Der Versuch eignet sich gut, um mit verhältnismäßig einfachen Mitteln, die Lebensdauer von Myonen zu bestimmen.

%Kurze Zusammenfassung der Ergebnisse
%-Vergleich mit Literaturwerten
%-Vergleich mit verschiedenen Messverfahren
%-bei Abweichungen mögliche Ursachen finden