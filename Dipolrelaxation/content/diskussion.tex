\section{Diskussion}
\label{sec:Diskussion}


In Tabelle \ref{tab:zusa} sind die bestimmten Aktivierungsenergien zusammenfassend dargestellt. Laut \cite{Dipol} ist der Literaturwert auf $W_\text{lit} = \SI{0.66}{\electronvolt}$ angegeben. Die jeweiligen Abweichungen sind ebenfalls der Tabelle zu entnehmen. 

\begin{table}
    \centering
    \caption{Zusammenfassung der Ergebnisse.}
    \label{tab:zusa}
    \sisetup{table-format = 1.2}
    \begin{tabular}{S S S S S}
        \toprule
        & \multicolumn{2}{c}{Integration} & \multicolumn{2}{c}{Approximation} \\
        \cmidrule(lr){2-3} \cmidrule(lr){4-5}
        {$\text{mittl. Heizrate} \mathbin{/} \si{\kelvin\per\minute} $} & {$W \mathbin{/} \si{\electronvolt}$} & {$\text{Abw.} \mathbin{/} \si{\percent}$} & {$W \mathbin{/} \si{\electronvolt}$} & {$\text{Abw.} \mathbin{/} \si{\percent}$} \\
        \midrule   
        1.36 & \num{0.91(04)} & \num{38(6)} & \num{0.674(030)} & \num{2(5)}  \\
        1.91 & \num{0.709(023)} & \num{7.4(35)} & \num{0.893(023)} & \num{35.5(35)}  \\
        \bottomrule
    \end{tabular}
\end{table}

Es fällt auf, dass keine der beiden Methoden oder der beiden Heizraten ein eindeutig besseres Ergebnis liefern. Scheinbar ist aber die Integrations-Methode bei einer höheren Heizrate und die Approximations-Methode bei einer niedrigeren Heizrate genauer. Da aber nur zwei verschiedenen Heizraten verwendet werden, ist diese Aussage nur als Vermutung zu verstehen.


Die berechneten Relaxationszeiten weichen mehrere Größenordungen von dem Literaturwert von $\tau_{0,\text{lit}} = \SI{4e-14}{\s}$ \cite{Dipol} ab. 
Da die Aktivierungsenergien exponetiell in die berechnung der Relaxationszeiten eingehen, sorgen bereits kleine Abweichungen für große Abweichungen der bestimmten Relaxationszeiten. 


Bei Betrachtung der Kurvenverläufe des Strom gegenüber der Temperatur fällt ein nachfolgendes Maximum auf. Innerhalb der hier präsentierten Theorie lassen sich diese Verläufe nicht erklären. Jedoch werden zum herleiten der Gleichungen nur die Dipolmomente erster Ordnung betrachtet. Diese zweiten Maxima werden durch Momente höherer Ordnung verursacht. Durch die steigende Temperatur können so Dipole ralaxieren, die eine höhere Coulomb-Barriere überwinden und damit eine höhere Aktivierungsenergie aufbringen müssen. Außerdem kann dieser Kurvenverlauf durch noch verbleibendes Wasser auf dem Kristall verursacht werden. 


Da im Versuch nur mit einer Art Kristall gearbeitet wird, kann kaum auf die Güte des Versuchs im Allgemeinen geschlossen werden. In diesem konkreten Fall aber, kann die Relaxation gut gezeigt werden. Auch eine grobe Bestimmung der Aktivierungsenergien ist möglich. Für eine gute Bestimmung der Relaxationszeiten ist jedoch die Präzision nicht hoch genug.
%Kurze Zusammenfassung der Ergebnisse
%-Vergleich mit Literaturwerten
%-Vergleich mit verschiedenen Messverfahren
%-bei Abweichungen mögliche Ursachen finden