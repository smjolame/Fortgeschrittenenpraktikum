\section{Diskussion}
\label{sec:Diskussion}

\subsection{TEM-Moden}

Die Verläufe in den Abbildungen [] zeigen für die $\text{TEM}_{00}$-Mode, wie auch für die $\text{TEM}_{01}$-Mode jeweils
die charakteristische Form der Mode, wie sie zu erwarten ist. Die Ausgleichsfunktionen zeigen somit nur kleine Abweichungen von den 
eigentlichen Messwerten. Eine mögliche Fehlerquelle stellt die Messung des Photostroms dar, da die Werte beim Ablesen teils starke und 
unregelmäßige Schwankungen aufwiesen. Desweiteren ist bei beiden Moden zu sehen, dass die Intensitätswerte für positive $l$ im Schnitt etwas 
größer sind. Das Ablesen der Werte wurde für die Durchführung aufgeteilt, wobei später angemerkt wurde, dass von den ablesenden Personen 
einer einen weißen, und einer einen schwarzen Pullover trägt. Da die ablesende Person immer in der Nähe der Photodiode steht, könnte das dadurch
entstehende unterschiedlich relfektierende Umfeld der Photodiode eine Abweichung des abzulesenden Photostroms herbeiführen. 

\subsection{Polarisation des Lasers}

Anhand der vorliegenden Messwerte in Abbildung [] ist zu sehen, dass die Intensität wie erwartet $2\pi$-periodisch verläuft. Es sind bei dieser 
Messung keine weiteren Auffälligkeiten zu sehen. 

\subsection{Wellenlänge}

Die gemessene Wellenlänge des Lasers beträgt $\SI{652(16)}{\nano\m}$, was um etwa $\SI{3}{\percent}$ vom Theoriewert $\SI{632.8}{\nano\m}$ 
abweicht. Da die Messung hierfür per Maßband und Augenmaß erfolgte, kann diese Abweichung durchaus Folge fehlender Präzision bei der Messung sein. 
Insgesamt ist eine Abweichung von $\SI{3}{\percent}$ in diesem Fall jedoch ein akzeptables Ergebnis. 