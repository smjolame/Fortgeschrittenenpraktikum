\section{Theorie}
\label{sec:Theorie}

Myonen entstehen in der Hochatmosphäre, wobei zuerst durch Wechselwirkung 
hochenergetischer Protonen Pionen entstehen, welche eine geringe Lebensdauer aufweisen und deshalb praktisch sofort weiter zerfallen, wobei Myonen und Myon-Neutrinos 
wie hier gezeigt entstehen:

\begin{equation}
    \pi^+ \rightarrow \nu_{\mu} \. ; \. \pi^- \rightarrow \overline{\nu}_{\mu}
\end{equation}

Diese Myonen bewegen sich dann in Richtung Erdoberfläche und können da mit einem Szintillatordetektor gemessen werden. Das hierbei zugrundeliegende Konzept ist, dass die in den 
Szintillator eintretenden Myonen kinetische Energie an das Szintillatormaterial abgeben. Dadurch werden die Molekühle im Tank vorübergehend angeregt und emittieren beim Zurückfallen 
auf den Grundzustand ein Photon. Diese Photonen werden dann mit einem Photomultiplier gemessen. Zerfällt das Myon innerhalb des Szintillators, wird ein weiteres Signal gemessen, da 
beim Zerfall des Myons ein Elektron entsteht, welches ebenfalls geladen ist. Ist das eintretende Myon negativ geladen, kann im Szintillatormaterial ein myonisches Atom entstehen, 
welches durch Absorbtion des Myons an einer freien Stelle im Atom zustande kommt. Das eintretende muss auch ohne diesen Effekt nicht zwingend innerhalb des Szintillators 
zerfallen. Besitzt es beim Eintreten noch genügend Energie, so kann es sich durch den kompletten Szintillator bewegen, ohne zu zerfallen und folglich ohne ein weiteres Signal zu 
erzeugen. 

Für die Berechnung der Lebensdauer ist der proportionale Zusammenhang 

\begin{equation}
    \text{d}W = \lambda \text{d}t
\end{equation}

essentiell, wobei $\text{d}W$ die Zerfallswahrscheinlichkeit im infinitesimalen Bereich und $\lambda$ eine charakteristische Zeitkonstante darstellt. Die Lebensdauer eines Myons ist also 
nicht von seinem Alter abhängig. Bei $N$ betrachteten Teilchen ist die Anzahl der im Zeitraum $\text{d}t$ zerfallenen Teilchen durch $\text{d}N$ auszudrücken, die sich durch 

\begin{equation}
    \text{d}N = -N\text{d}W = -\lambda N\text{d}t 
\end{equation}

auszudrücken lässt. Durch Integration nach der Zeit ergibt sich 

\begin{equation}
    \frac{N(t)}{N_0} = e^{-\lambda t}
\end{equation}

mit $N_0$ als Anzahl der insgesamt betrachteten Teilchen und $\lambda$ als Zerfallskonstante. Die Lebensdauer ist also auf dem Intervall $t$ bis $\text{d}t$ exponentialverteilt:

\begin{equation}
    \text{d}N(t) = N_0 \lambda e^{-\lambda t} \text{d}t
\end{equation}

Die charakteristische Lebensdauer lässt sich dann durch Bildung des Mittelwertes aller möglichen Lebensdauern identifizieren: 

\begin{equation}
    <t> = \tau = \int_0^{\infty} \lambda e^{-\lambda t} \text{d}t = \frac{1}{\lambda}
\end{equation}